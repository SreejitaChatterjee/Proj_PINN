% DSN 2026 Submission: Breaking the Residual Barrier
% IEEE Conference format

\documentclass[conference]{IEEEtran}

\usepackage{amsmath,amssymb,amsfonts}
\usepackage{algorithmic}
\usepackage{graphicx}
\usepackage{textcomp}
\usepackage{xcolor}
\usepackage{booktabs}
\usepackage{multirow}

% Comments for drafting
\newcommand{\todo}[1]{\textcolor{red}{[TODO: #1]}}
\newcommand{\note}[1]{\textcolor{blue}{[Note: #1]}}

\begin{document}

\title{Breaking the Residual Barrier: Detecting GPS Spoofing\\via Inverse-Cycle Instability}

\author{
\IEEEauthorblockN{Anonymous Author(s)}
\IEEEauthorblockA{Anonymous Institution}
}

\maketitle

% ==============================================================================
\begin{abstract}
% ==============================================================================

Residual-based anomaly detection is the dominant approach for GPS spoofing detection in autonomous systems. We prove this approach has a fundamental blind spot: \textbf{consistent spoofing}---attacks that preserve dynamics relationships---produces identical residual distributions to nominal flight, making detection impossible regardless of model accuracy.

We formalize this limitation through \textbf{Residual Equivalence Classes (RECs)} and demonstrate that consistent GPS spoofing lies within the nominal REC by construction (residual AUROC = 0.500).

To break this barrier, we introduce \textbf{Inverse-Cycle Instability (ICI)}, a bidirectional consistency test that exploits a fundamental asymmetry: while attackers can enforce forward consistency, they cannot preserve inverse-cycle consistency without access to the learned inverse dynamics. ICI achieves AUROC = 1.000 on attacks provably invisible to residual-based methods, with detection signal scaling monotonically with spoof magnitude.

\textbf{Key insight:} Forward consistency is cheap. Inverse consistency is expensive. That asymmetry enables detection.

\end{abstract}

\begin{IEEEkeywords}
GPS spoofing, anomaly detection, cyber-physical systems, inverse dynamics, cycle consistency
\end{IEEEkeywords}

% ==============================================================================
\section{Introduction}
% ==============================================================================

GPS spoofing poses a critical threat to autonomous systems. The dominant defense is \textit{residual-based anomaly detection}: learn a dynamics model $f_\theta$, compute prediction residuals $\|x_{t+1} - f_\theta(x_t)\|$, and flag anomalies when residuals exceed a threshold.

\textbf{The problem:} This approach assumes attacks induce detectable residuals. We prove this assumption fails for \textit{consistent spoofing}---attacks that preserve the dynamics relationship between consecutive states.

\subsection{Contributions}

\begin{enumerate}
    \item \textbf{Inverse-Cycle Instability (ICI):} A new detection primitive that breaks the residual barrier without external sensors (Section~\ref{sec:ici})

    \item \textbf{Residual Equivalence Classes (RECs):} A formal framework proving when residual-based detection is fundamentally impossible (Section~\ref{sec:rec})

    \item \textbf{Impossibility Result:} Empirical verification that consistent GPS spoofing lies within the nominal REC (delta difference = 0.0, AUROC = 0.500)

    \item \textbf{Empirical Validation:} ICI achieves AUROC = 1.000 on attacks invisible to residuals, with monotonically increasing signal (Section~\ref{sec:eval})
\end{enumerate}

% ==============================================================================
\section{Residual Equivalence Classes}
\label{sec:rec}
% ==============================================================================

\begin{definition}[Residual Equivalence Class]
Given a learned dynamics model $f_\theta$, two trajectories $\{x_t\}$ and $\{\tilde{x}_t\}$ belong to the same \textbf{Residual Equivalence Class (REC)} if they induce statistically indistinguishable prediction residual distributions:
\begin{equation}
\|x_{t+1} - f_\theta(x_t)\| \stackrel{d}{=} \|\tilde{x}_{t+1} - f_\theta(\tilde{x}_t)\|
\end{equation}
\end{definition}

\textbf{Implication:} Any detector operating solely on residual statistics cannot distinguish trajectories within the same REC.

\subsection{Consistent Spoofing is Undetectable}

\begin{theorem}[Impossibility]
Let $\{x_t\}$ be a nominal trajectory and $\{\tilde{x}_t\} = \{x_t + c\}$ be a consistently spoofed trajectory with constant offset $c$. If $f_\theta$ is translation-equivariant in its residuals (i.e., predicts state \textit{changes}), then $\{x_t\}$ and $\{\tilde{x}_t\}$ belong to the same REC.
\end{theorem}

\begin{proof}
For translation-equivariant dynamics:
\begin{align}
\tilde{x}_{t+1} - f_\theta(\tilde{x}_t) &= (x_{t+1} + c) - (f_\theta(x_t) + c) \\
&= x_{t+1} - f_\theta(x_t)
\end{align}
The residuals are identical, thus indistinguishable.
\end{proof}

\subsection{Empirical Verification}

We verify this impossibility empirically:

\begin{table}[h]
\centering
\caption{REC Membership Verification}
\begin{tabular}{lcc}
\toprule
Trajectory & Delta Difference & AUROC \\
\midrule
Nominal vs Nominal & 0.0 & 0.500 \\
Nominal vs 100m offset & 0.0 & 0.500 \\
Nominal vs Growing drift & $>0$ & 0.845 \\
\bottomrule
\end{tabular}
\end{table}

Consistent spoofing produces \textit{identical} delta residuals (difference = 0.0), confirming the impossibility result.

% ==============================================================================
\section{Inverse-Cycle Instability}
\label{sec:ici}
% ==============================================================================

\subsection{Core Insight}

Residual-based detection tests \textit{forward} consistency:
\begin{equation}
x_{t+1} \approx f_\theta(x_t) \quad \text{(spoofable)}
\end{equation}

An attacker controlling GPS observations can ensure this constraint holds.

ICI tests \textit{bidirectional} cycle consistency:
\begin{equation}
x_t \approx g_\phi(f_\theta(x_t)) \quad \text{(not spoofable)}
\end{equation}

where $g_\phi$ is a learned inverse dynamics model.

\textbf{Key asymmetry:} Forward consistency is cheap. Inverse consistency is expensive. The attacker cannot preserve cycle consistency without knowledge of $g_\phi$.

\subsection{Why ICI Works}

\begin{enumerate}
    \item Nominal trajectories lie on a stable forward-inverse manifold
    \item The inverse model $g_\phi$ is trained only on nominal data
    \item Spoofed trajectories (even consistent ones) lie off-manifold
    \item Off-manifold inputs produce high cycle reconstruction error
\end{enumerate}

\subsection{ICI Score}

\begin{equation}
\text{ICI}_t = \|x_t - g_\phi(f_\theta(x_t))\|
\end{equation}

\subsection{Training Protocol}

We train $f_\theta$ and $g_\phi$ sequentially:

\textbf{Phase 1:} Train forward model $f_\theta$
\begin{equation}
\mathcal{L}_{\text{forward}} = \mathbb{E}[\|x_{t+1} - f_\theta(x_t)\|^2]
\end{equation}

\textbf{Phase 2:} Freeze $f_\theta$, train inverse model $g_\phi$
\begin{equation}
\mathcal{L} = \mathcal{L}_{\text{inv}} + \lambda \mathcal{L}_{\text{cycle}}
\end{equation}
where:
\begin{align}
\mathcal{L}_{\text{inv}} &= \mathbb{E}[\|x_t - g_\phi(x_{t+1})\|^2] \\
\mathcal{L}_{\text{cycle}} &= \mathbb{E}[\|x_t - g_\phi(f_\theta(x_t))\|^2]
\end{align}

We use $\lambda = 0.25$ to anchor the inverse to the learned forward manifold.

% ==============================================================================
\section{Evaluation}
\label{sec:eval}
% ==============================================================================

\subsection{Impossibility Demonstration}

\begin{table}[h]
\centering
\caption{ICI Breaks the Residual Barrier}
\begin{tabular}{lccc}
\toprule
Method & AUROC & Recall@1\%FPR & Status \\
\midrule
Delta Residual & 0.500 & N/A & Same REC \\
\textbf{ICI (ours)} & \textbf{1.000} & \textbf{1.000} & Breaks barrier \\
\bottomrule
\end{tabular}
\end{table}

On 100m consistent spoofing: residual deltas are \textit{identical} (diff = 0.0), but ICI detects perfectly.

\subsection{Sensitivity Analysis}

ICI provides a graded signal proportional to structural deviation:

\begin{table}[h]
\centering
\caption{ICI Sensitivity to Spoof Magnitude}
\begin{tabular}{rrrr}
\toprule
Offset (m) & ICI Mean & Ratio & AUROC \\
\midrule
0 & 3.4 & 1.0x & 0.50 \\
5 & 8.2 & 2.4x & 1.00 \\
10 & 13.7 & 4.0x & 1.00 \\
25 & 30.8 & 8.9x & 1.00 \\
50 & 59.4 & 17.2x & 1.00 \\
100 & 116.6 & 33.9x & 1.00 \\
200 & 231.2 & 67.1x & 1.00 \\
\bottomrule
\end{tabular}
\end{table}

\textbf{Monotonic increase confirmed.} This is not a binary trick---ICI scales linearly with displacement ($\approx$1.2x per meter).

\subsection{Why AUROC = 1.0 is Believable}

Perfect separation arises because consistent spoofing induces a \textit{deterministic} off-manifold shift rather than a stochastic perturbation. The inverse model is only accurate on the learned state manifold; off-manifold inputs produce cycle errors that scale with displacement.

% ==============================================================================
\section{Related Work}
\label{sec:related}
% ==============================================================================

\textbf{GPS Spoofing Detection.} Prior work uses residual-based detection with Kalman filters \cite{}, physics-based consistency checks \cite{}, and machine learning \cite{}. All assume attacks induce detectable residuals. We prove this assumption fails for consistent spoofing.

\textbf{Cycle Consistency.} Cycle-consistency losses are used in image-to-image translation (CycleGAN \cite{}) and self-supervised learning. To our knowledge, we are the first to apply cycle consistency as a \textit{security signal} for CPS anomaly detection.

\textbf{Inverse Dynamics.} Inverse dynamics models are used in robotics for control \cite{}. We use them for \textit{detection}, exploiting the instability of off-manifold inputs.

% ==============================================================================
\section{Discussion}
\label{sec:discussion}
% ==============================================================================

\subsection{Threat Model}

ICI assumes the attacker:
\begin{itemize}
    \item Controls GPS observations
    \item Knows the forward model architecture (but not weights)
    \item Does \textit{not} know the inverse model $g_\phi$
\end{itemize}

If the attacker learns $g_\phi$, they could potentially craft cycle-consistent spoofing. Defense: periodically retrain $g_\phi$ with fresh randomness.

\subsection{Limitations}

\begin{itemize}
    \item ICI requires training an inverse model (additional computation)
    \item Perfect detection on synthetic data; real-world evaluation ongoing
    \item Assumes dynamics are learnable and invertible
\end{itemize}

\subsection{Scope}

\begin{itemize}
    \item No additional sensors required
    \item Nominal-only training (no attack examples needed)
    \item Real-time feasible ($<$5ms inference at 200Hz)
\end{itemize}

% ==============================================================================
\section{Conclusion}
% ==============================================================================

We introduced Inverse-Cycle Instability (ICI), a new detection primitive for GPS spoofing that breaks the fundamental limitation of residual-based methods. By formalizing detectability through Residual Equivalence Classes, we proved that consistent spoofing is invisible to residuals (AUROC = 0.500) and showed ICI achieves perfect detection (AUROC = 1.000) by exploiting the asymmetry between forward and inverse consistency.

\textbf{Key insight:} Forward consistency is cheap. Inverse consistency is expensive. That asymmetry is the breakthrough.

% ==============================================================================
% References (to be filled)
% ==============================================================================
\bibliographystyle{IEEEtran}
% \bibliography{references}

\end{document}
