\documentclass[12pt,a4paper]{article}
\usepackage[utf8]{inputenc}
\usepackage[T1]{fontenc}
\usepackage{lmodern}
\usepackage{amsmath,amsfonts,amssymb}
\usepackage{booktabs}
\usepackage{longtable}
\usepackage{array}
\usepackage{geometry}
\usepackage{xcolor}

\geometry{margin=1in}

\title{\textbf{PINN Physics Layer Reference}}
\author{}
\date{}

\begin{document}

\maketitle

\section*{Physics Laws \& Equations Used in PINN Physics Layer}

All physics equations are based on established rigid body dynamics and multirotor control literature \cite{mahony2012multirotor,beard2012small}.

\subsection*{1. Rotational Dynamics (Euler's Equations)}

\textbf{Equation 1.1 - Roll Angular Acceleration:}
\begin{equation}
\dot{p} = \frac{J_{yy} - J_{zz}}{J_{xx}} \cdot q \cdot r + \frac{\tau_x}{J_{xx}}
\end{equation}
where:
\begin{itemize}
\item $p$ = roll rate (rad/s)
\item $\dot{p}$ = roll angular acceleration (rad/s²)
\item $q$ = pitch rate (rad/s)
\item $r$ = yaw rate (rad/s)
\item $\tau_x$ = roll torque (N·m)
\item $J_{xx}$ = moment of inertia about x-axis = $6.86 \times 10^{-5}$ kg·m²
\item $J_{yy}$ = moment of inertia about y-axis = $9.2 \times 10^{-5}$ kg·m²
\item $J_{zz}$ = moment of inertia about z-axis = $1.366 \times 10^{-4}$ kg·m²
\end{itemize}

\textbf{Equation 1.2 - Pitch Angular Acceleration:}
\begin{equation}
\dot{q} = \frac{J_{zz} - J_{xx}}{J_{yy}} \cdot p \cdot r + \frac{\tau_y}{J_{yy}}
\end{equation}
where:
\begin{itemize}
\item $q$ = pitch rate (rad/s)
\item $\dot{q}$ = pitch angular acceleration (rad/s²)
\item $\tau_y$ = pitch torque (N·m)
\end{itemize}

\textbf{Equation 1.3 - Yaw Angular Acceleration:}
\begin{equation}
\dot{r} = \frac{J_{xx} - J_{yy}}{J_{zz}} \cdot p \cdot q + \frac{\tau_z}{J_{zz}}
\end{equation}
where:
\begin{itemize}
\item $r$ = yaw rate (rad/s)
\item $\dot{r}$ = yaw angular acceleration (rad/s²)
\item $\tau_z$ = yaw torque (N·m)
\end{itemize}

\textbf{Key Feature:} Real Euler equations with NO artificial damping terms. Pure physics-based rotational dynamics \cite{mahony2012multirotor}.

\subsection*{2. Simplified Euler Angle Integration}

\textbf{Equation 2.1 - Roll Angle Rate:}
\begin{equation}
\dot{\phi} = p
\end{equation}
where:
\begin{itemize}
\item $\phi$ = roll angle (rad)
\item $\dot{\phi}$ = roll angle rate (rad/s)
\item $p$ = roll rate (rad/s)
\end{itemize}

\textbf{Equation 2.2 - Pitch Angle Rate:}
\begin{equation}
\dot{\theta} = q
\end{equation}
where:
\begin{itemize}
\item $\theta$ = pitch angle (rad)
\item $\dot{\theta}$ = pitch angle rate (rad/s)
\item $q$ = pitch rate (rad/s)
\end{itemize}

\textbf{Equation 2.3 - Yaw Angle Rate:}
\begin{equation}
\dot{\psi} = r
\end{equation}
where:
\begin{itemize}
\item $\psi$ = yaw angle (rad)
\item $\dot{\psi}$ = yaw angle rate (rad/s)
\item $r$ = yaw rate (rad/s)
\end{itemize}

\textbf{Note:} These are simplified angular integrations valid for small angles, not the full nonlinear Euler kinematics \cite{beard2012small}.

\subsection*{3. Translational Dynamics (Vertical Motion)}

\textbf{Equation 3.1 - Vertical Acceleration:}
\begin{equation}
\dot{v}_z = -g + \frac{T}{m \cdot \cos(\theta) \cdot \cos(\phi)}
\end{equation}
where:
\begin{itemize}
\item $v_z$ = vertical velocity (m/s, positive downward in NED frame)
\item $\dot{v}_z$ = vertical acceleration (m/s²)
\item $T$ = total thrust (N)
\item $\theta$ = pitch angle (rad)
\item $\phi$ = roll angle (rad)
\item $m$ = quadrotor mass = 0.068 kg
\item $g$ = gravitational acceleration = 9.81 m/s²
\end{itemize}

\textbf{Equation 3.2 - Altitude Rate:}
\begin{equation}
\dot{z} = v_z
\end{equation}
where:
\begin{itemize}
\item $z$ = altitude (m, positive downward in NED frame)
\item $\dot{z}$ = altitude rate (m/s)
\item $v_z$ = vertical velocity (m/s)
\end{itemize}

\textbf{Note:} No aerodynamic drag is included in the physics loss formulation. Translational dynamics based on Newton's second law \cite{mahony2012multirotor,beard2012small}.

\newpage

\section*{Equations Used in Data Generation but NOT in PINN}

\subsection*{4. Full Nonlinear Euler Kinematics (Data Generation Only)}

\textbf{Equation 4.1 - Full Roll Angle Rate:}
\begin{equation}
\dot{\phi} = p + \sin(\phi)\tan(\theta) \cdot q + \cos(\phi)\tan(\theta) \cdot r
\end{equation}

\textbf{Equation 4.2 - Full Pitch Angle Rate:}
\begin{equation}
\dot{\theta} = \cos(\phi) \cdot q - \sin(\phi) \cdot r
\end{equation}

\textbf{Equation 4.3 - Full Yaw Angle Rate:}
\begin{equation}
\dot{\psi} = \frac{\sin(\phi) \cdot q + \cos(\phi) \cdot r}{\cos(\theta)}
\end{equation}

\textbf{Why not in PINN:} PINN uses simplified small-angle approximations ($\dot{\phi} = p$, $\dot{\theta} = q$, $\dot{\psi} = r$) which are valid for typical quadrotor maneuvers with angles $< 30°$ \cite{beard2012small}.

\subsection*{5. Full 6DOF Translational Dynamics (Data Generation Only)}

\textbf{Equation 5.1 - X-Velocity Acceleration:}
\begin{equation}
\dot{u} = r \cdot v - q \cdot w + \frac{f_x}{m} - g\sin(\theta) - c_d \cdot u \cdot |u|
\end{equation}

\textbf{Equation 5.2 - Y-Velocity Acceleration:}
\begin{equation}
\dot{v} = p \cdot w - r \cdot u + \frac{f_y}{m} + g\cos(\theta)\sin(\phi) - c_d \cdot v \cdot |v|
\end{equation}

\textbf{Equation 5.3 - Z-Velocity Acceleration (with drag):}
\begin{equation}
\dot{w} = q \cdot u - p \cdot v + \frac{f_z}{m} + g\cos(\theta)\cos(\phi) - c_d \cdot w \cdot |w|
\end{equation}

where:
\begin{itemize}
\item $u, v, w$ = body-frame velocities in x, y, z directions (m/s)
\item $f_x, f_y, f_z$ = body-frame forces (N)
\item $c_d = 0.05$ kg/m = quadratic drag coefficient
\item Coriolis terms: $r \cdot v - q \cdot w$, $p \cdot w - r \cdot u$, $q \cdot u - p \cdot v$
\end{itemize}

\textbf{Why not in PINN:} PINN models only vertical (z-axis) dynamics for altitude control. Horizontal motion ($u, v$) is not controlled in the training data, and the simplified model assumes $f_x = f_y = 0$. Additionally, drag forces are omitted from physics loss. Full 6-DOF dynamics from \cite{mahony2012multirotor,beard2012small}.

\subsection*{6. Body-to-World Frame Transformation (Data Generation Only)}

\textbf{Equation 6.1 - X-Position Rate:}
\begin{multline}
\dot{x} = \cos(\psi)\cos(\theta) \cdot u + [\cos(\psi)\sin(\theta)\sin(\phi) - \sin(\psi)\cos(\phi)] \cdot v \\
+ [\sin(\psi)\sin(\phi) + \cos(\psi)\sin(\theta)\cos(\phi)] \cdot w
\end{multline}

\textbf{Equation 6.2 - Y-Position Rate:}
\begin{multline}
\dot{y} = \sin(\psi)\cos(\theta) \cdot u + [\cos(\psi)\cos(\phi) + \sin(\psi)\sin(\theta)\sin(\phi)] \cdot v \\
+ [\sin(\psi)\sin(\theta)\cos(\phi) - \cos(\psi)\sin(\phi)] \cdot w
\end{multline}

\textbf{Equation 6.3 - Z-Position Rate:}
\begin{equation}
\dot{z} = -[\sin(\theta) \cdot u - \cos(\theta)\sin(\phi) \cdot v - \cos(\theta)\cos(\phi) \cdot w]
\end{equation}

\textbf{Why not in PINN:} PINN focuses on body-frame dynamics, not world-frame position tracking. Position is not required for learning the dynamics model. Rotation matrix formulation from \cite{mahony2012multirotor}.

\subsection*{7. Motor Thrust/Torque Mapping (Not Currently Used)}

\textbf{Equation 7.1 - Total Thrust from Motors:}
\begin{equation}
T = k_t \cdot (\omega_1^2 + \omega_2^2 + \omega_3^2 + \omega_4^2)
\end{equation}

\textbf{Equation 7.2 - Roll Torque from Motors:}
\begin{equation}
\tau_x = k_t \cdot L \cdot (\omega_2^2 - \omega_4^2)
\end{equation}

\textbf{Equation 7.3 - Pitch Torque from Motors:}
\begin{equation}
\tau_y = k_t \cdot L \cdot (\omega_3^2 - \omega_1^2)
\end{equation}

\textbf{Equation 7.4 - Yaw Torque from Motors:}
\begin{equation}
\tau_z = k_q \cdot (\omega_1^2 - \omega_2^2 + \omega_3^2 - \omega_4^2)
\end{equation}

where:
\begin{itemize}
\item $\omega_1, \omega_2, \omega_3, \omega_4$ = motor speeds (rad/s)
\item $k_t = 0.01$ N/(rad/s)² = thrust coefficient
\item $k_q = 7.8263 \times 10^{-4}$ N·m/(rad/s)² = torque coefficient
\item $L$ = arm length from center to motor (m)
\end{itemize}

\textbf{Why not in PINN:} Currently, PINN receives thrust $T$ and torques $\tau_x, \tau_y, \tau_z$ as direct inputs rather than computing them from motor speeds. These equations are only used to calculate physical limits during data generation. Motor thrust/torque coefficients from \cite{mahony2012multirotor}.

\newpage

\section*{Complete Mapping: 18 State Outputs to Equations}

\subsection*{8. State Variables and Their Physics Equations}

\begin{longtable}{|p{0.04\textwidth}|p{0.12\textwidth}|p{0.27\textwidth}|p{0.13\textwidth}|p{0.28\textwidth}|}
\hline
\textbf{\#} & \textbf{Output} & \textbf{Equation Used} & \textbf{Status} & \textbf{Reason/Note} \\
\hline
\endhead

\multicolumn{5}{|c|}{\textbf{Rotational Dynamics}} \\
\hline
1 & $\dot{p}$ & Eq 1.1: $\dot{p} = \frac{J_{yy} - J_{zz}}{J_{xx}} \cdot q \cdot r + \frac{\tau_x}{J_{xx}}$ & \textcolor{blue}{USED} & Full Euler equation \\
\hline
2 & $\dot{q}$ & Eq 1.2: $\dot{q} = \frac{J_{zz} - J_{xx}}{J_{yy}} \cdot p \cdot r + \frac{\tau_y}{J_{yy}}$ & \textcolor{blue}{USED} & Full Euler equation \\
\hline
3 & $\dot{r}$ & Eq 1.3: $\dot{r} = \frac{J_{xx} - J_{yy}}{J_{zz}} \cdot p \cdot q + \frac{\tau_z}{J_{zz}}$ & \textcolor{blue}{USED} & Full Euler equation \\
\hline
4 & $p$ & Integration: $p_{t+dt} = p_t + \dot{p} \cdot dt$ & \textcolor{blue}{USED} & Direct integration \\
\hline
5 & $q$ & Integration: $q_{t+dt} = q_t + \dot{q} \cdot dt$ & \textcolor{blue}{USED} & Direct integration \\
\hline
6 & $r$ & Integration: $r_{t+dt} = r_t + \dot{r} \cdot dt$ & \textcolor{blue}{USED} & Direct integration \\
\hline

\multicolumn{5}{|c|}{\textbf{Euler Angle Kinematics}} \\
\hline
7 & $\dot{\phi}$ & Eq 2.1 (PINN): $\dot{\phi} = p$ \newline Eq 4.1 (Data): Full nonlinear & \textcolor{orange}{SIMPLIFIED} & Small-angle approximation valid for $|\phi| < 30°$. Data uses full kinematics with $\sin, \tan$ terms. \\
\hline
8 & $\dot{\theta}$ & Eq 2.2 (PINN): $\dot{\theta} = q$ \newline Eq 4.2 (Data): Full nonlinear & \textcolor{orange}{SIMPLIFIED} & Small-angle approximation valid for $|\theta| < 30°$. Data uses full kinematics with $\cos, \sin$ terms. \\
\hline
9 & $\dot{\psi}$ & Eq 2.3 (PINN): $\dot{\psi} = r$ \newline Eq 4.3 (Data): Full nonlinear & \textcolor{orange}{SIMPLIFIED} & Small-angle approximation. Data uses full kinematics with gimbal lock at $\theta = \pm 90°$. \\
\hline
10 & $\phi$ & Integration: $\phi_{t+dt} = \phi_t + \dot{\phi} \cdot dt$ & \textcolor{blue}{USED} & Uses simplified $\dot{\phi}$ \\
\hline
11 & $\theta$ & Integration: $\theta_{t+dt} = \theta_t + \dot{\theta} \cdot dt$ & \textcolor{blue}{USED} & Uses simplified $\dot{\theta}$ \\
\hline
12 & $\psi$ & Integration: $\psi_{t+dt} = \psi_t + \dot{\psi} \cdot dt$ & \textcolor{blue}{USED} & Uses simplified $\dot{\psi}$ \\
\hline

\multicolumn{5}{|c|}{\textbf{Translational Dynamics}} \\
\hline
13 & $\dot{u}$ & Eq 5.1: Full 6DOF with Coriolis and drag & \textcolor{red}{NOT USED} & PINN focuses on vertical-only control. No horizontal force inputs ($f_x = 0$) in training data. \\
\hline
14 & $\dot{v}$ & Eq 5.2: Full 6DOF with Coriolis and drag & \textcolor{red}{NOT USED} & PINN focuses on vertical-only control. No horizontal force inputs ($f_y = 0$) in training data. \\
\hline
15 & $\dot{w}$ & Eq 3.1 (PINN): No drag \newline Eq 5.3 (Data): With drag & \textcolor{orange}{SIMPLIFIED} & PINN omits Coriolis terms (assumes $u=v=0$) and drag ($c_d \cdot w \cdot |w|$) for simpler vertical dynamics. \\
\hline
16 & $u$ & Integration of $\dot{u}$ & \textcolor{red}{NOT USED} & Not predicted since $\dot{u}$ is not modeled. Passive/uncontrolled horizontal motion. \\
\hline
17 & $v$ & Integration of $\dot{v}$ & \textcolor{red}{NOT USED} & Not predicted since $\dot{v}$ is not modeled. Passive/uncontrolled horizontal motion. \\
\hline
18 & $w$ ($v_z$) & Integration: $w_{t+dt} = w_t + \dot{w} \cdot dt$ & \textcolor{blue}{USED} & Uses simplified $\dot{w}$ \\
\hline
\end{longtable}

\subsection*{Legend}
\begin{itemize}
\item \textcolor{blue}{USED} - Equation actively used in PINN physics loss (perfect match with data generation)
\item \textcolor{orange}{SIMPLIFIED} - PINN uses a simplified approximation; data generation uses full complex equation
\item \textcolor{red}{NOT USED} - Variable exists in training data but completely absent from PINN physics loss
\end{itemize}

\subsection*{Detailed Explanations}

\textbf{Why SIMPLIFIED (4 variables):}
\begin{itemize}
\item \textbf{Outputs 7-9 (Euler angle rates):} PINN uses small-angle approximations ($\dot{\phi} = p$, $\dot{\theta} = q$, $\dot{\psi} = r$) which are valid for typical quadrotor maneuvers with angles $< 30°$. Data generation uses full nonlinear Euler kinematics with trigonometric coupling terms ($\sin, \cos, \tan$).
\item \textbf{Output 15 (vertical acceleration):} PINN omits Coriolis coupling terms (assumes horizontal velocities $u=v=0$) and aerodynamic drag ($c_d \cdot w \cdot |w|$) for a simpler vertical-only dynamics model.
\end{itemize}

\textbf{Why NOT USED (4 variables):}
\begin{itemize}
\item \textbf{Outputs 13-14 (horizontal accelerations):} PINN focuses on vertical-only quadrotor control. Training data has no horizontal control inputs ($f_x = f_y = 0$), making horizontal dynamics passive/uncontrolled. Not needed for altitude and attitude stabilization task.
\item \textbf{Outputs 16-17 (horizontal velocities):} Cannot be predicted since their derivatives ($\dot{u}, \dot{v}$) are not modeled in PINN. These are byproducts of the full 6DOF simulation but not required for the vertical control objective.
\end{itemize}

\subsection*{Summary of PINN Model Scope}

\textbf{What PINN Models:}
\begin{itemize}
\item Full 3D rotational dynamics (Euler's equations)
\item Simplified Euler angle integration (small-angle approximation)
\item Vertical translational dynamics (without drag)
\item Altitude tracking
\end{itemize}

\textbf{What PINN Does NOT Model:}
\begin{itemize}
\item Full nonlinear Euler kinematics
\item Horizontal motion (x, y velocities and positions)
\item Aerodynamic drag forces
\item Body-to-world frame transformations
\item Motor speed to thrust/torque mapping
\end{itemize}

\textbf{Justification:} The PINN focuses on a simplified vertical flight dynamics model suitable for altitude control and small-angle attitude stabilization, which covers the majority of typical quadrotor flight scenarios.

\newpage

\section*{References}

All physics equations in this document are sourced from the following established literature on quadrotor dynamics and control:

\begin{thebibliography}{9}

\bibitem{mahony2012multirotor}
Mahony, R., Kumar, V., \& Corke, P. (2012).
\textit{Multirotor Aerial Vehicles: Modeling, Estimation, and Control of Quadrotor}.
IEEE Robotics \& Automation Magazine, 19(3), 20-32.
DOI: 10.1109/MRA.2012.2206474
\\
\textbf{Coverage:} Complete Newton-Euler formulation for quadrotor 6-DOF dynamics including:
\begin{itemize}
\item Euler equations for rotational dynamics (Equations 1.1-1.3)
\item Full nonlinear Euler angle kinematics (Equations 4.1-4.3)
\item Translational dynamics in body frame (Equations 5.1-5.3)
\item Body-to-world frame rotation matrix transformations (Equations 6.1-6.3)
\item Motor thrust and torque coefficients (Equations 7.1-7.4)
\end{itemize}

\bibitem{beard2012small}
Beard, R. W., \& McLain, T. W. (2012).
\textit{Small Unmanned Aircraft: Theory and Practice}.
Princeton University Press. Chapter 4: Forces and Moments.
\\
\textbf{Coverage:} Forces and moments formulation for small UAVs including:
\begin{itemize}
\item Simplified Euler angle integration (Equations 2.1-2.3)
\item Small-angle approximation validity ranges
\item Vertical translational dynamics (Equation 3.1-3.2)
\item Aerodynamic drag modeling in body frame
\end{itemize}

\end{thebibliography}

\subsection*{Reference Coverage}

These two authoritative sources provide complete coverage of all physics equations used in both:
\begin{itemize}
\item \textbf{PINN Physics Loss:} Euler rotational dynamics, simplified kinematics, vertical dynamics
\item \textbf{Data Generation:} Full 6-DOF dynamics, nonlinear kinematics, frame transformations, motor mapping
\end{itemize}

The PINN implementation uses a subset of these equations (simplified model) while the data generation uses the complete formulation (full model), as documented in Section 8.

\end{document}
