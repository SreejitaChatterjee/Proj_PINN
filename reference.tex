\section{References}

The loss function architecture in this project combines multiple physics-based and data-driven components, drawing from established literature in Physics-Informed Neural Networks, quadrotor dynamics, and machine learning.

\subsection{Loss Function Components and Their Sources}

\subsubsection{1. Physics-Informed Neural Networks Framework}

The core PINN methodology, which combines data-driven learning with physical constraints through a composite loss function, is based on:

\begin{itemize}
\item \textbf{Raissi, M., Perdikaris, P., \& Karniadakis, G. E.} (2019). ``Physics-informed neural networks: A deep learning framework for solving forward and inverse problems involving nonlinear partial differential equations.'' \textit{Journal of Computational Physics}, 378, 686-707. DOI: 10.1016/j.jcp.2018.10.045
\end{itemize}

\textit{Coverage:} This seminal work introduces the physics loss term that enforces governing equations as soft constraints during training, enabling simultaneous parameter identification and state prediction.

\subsubsection{2. Newton-Euler Equations for Quadrotor Dynamics}

The physics loss implementation enforces complete 6-DOF rigid body dynamics (translational and rotational) with body-to-inertial frame transformations:

\begin{itemize}
\item \textbf{Beard, R. W., \& McLain, T. W.} (2012). \textit{Small Unmanned Aircraft: Theory and Practice}. Princeton University Press. Chapter 4: Forces and Moments.

\item \textbf{Mahony, R., Kumar, V., \& Corke, P.} (2012). ``Multirotor Aerial Vehicles: Modeling, Estimation, and Control of Quadrotor.'' \textit{IEEE Robotics \& Automation Magazine}, 19(3), 20-32. DOI: 10.1109/MRA.2012.2206474
\end{itemize}

\textit{Coverage:} These sources provide the complete Newton-Euler formulation including:
\begin{itemize}
\item Euler equations for rotational dynamics (angular acceleration from torques and gyroscopic effects)
\item Attitude kinematics (angular rate to Euler angle derivatives transformation)
\item Translational dynamics in body frame (linear acceleration from forces and Coriolis effects)
\item Position kinematics (body-to-inertial velocity transformation using rotation matrices)
\item Aerodynamic drag modeling (quadratic drag terms)
\end{itemize}

\subsubsection{3. Temporal Smoothness Constraints}

The temporal smoothness loss enforces physical limits on state derivatives (velocities and accelerations) to prevent unphysical jumps:

\begin{itemize}
\item \textbf{Karniadakis, G. E., Kevrekidis, I. G., Lu, L., Perdikaris, P., Wang, S., \& Yang, L.} (2021). ``Physics-informed machine learning.'' \textit{Nature Reviews Physics}, 3(6), 422-440. DOI: 10.1038/s42254-021-00314-5
\end{itemize}

\textit{Coverage:} Section on ``Temporal coherence'' discusses enforcing smoothness constraints on time derivatives to ensure physically realistic trajectories in dynamical systems. The use of ReLU-based soft constraints ($\text{ReLU}(|\dot{x}| - \dot{x}_{\text{max}})^2$) for enforcing bounds is a standard technique in physics-informed learning.

\subsubsection{4. Stability Loss and State Bounds}

The stability loss prevents state-space divergence by enforcing soft bounds on all state variables:

\begin{itemize}
\item \textbf{Mahony, Kumar, \& Corke} (2012) - same as reference [2] above.
\end{itemize}

\textit{Coverage:} Section 3.2 provides typical operating ranges for quadrotor states (attitude limits, angular rate bounds, velocity constraints) which justify the stability loss bounds. The bounded state-space approach ensures the neural network operates within the physical envelope of the quadrotor platform.

\subsubsection{5. Parameter Regularization}

The regularization loss penalizes deviations of learned physical parameters from their nominal values using normalized $L_2$ penalty:

\begin{itemize}
\item \textbf{Raissi et al.} (2019) - same as reference [1] above.
\end{itemize}

\textit{Coverage:} Section 3.3 on ``Discrete time models'' demonstrates the use of parameter regularization in PINNs to prevent overfitting and anchor learned parameters near physically plausible values. The normalized quadratic penalty $\sum_k \frac{(\theta_k - \theta_k^{\text{nom}})^2}{(\theta_k^{\text{nom}})^2}$ ensures scale-invariant regularization across parameters with different magnitudes.

\subsection{Summary of Coverage}

\begin{table}[H]
\centering
\begin{tabular}{lll}
\toprule
\textbf{Loss Component} & \textbf{Equation Elements} & \textbf{Primary Source} \\
\midrule
Data Loss & MSE between predictions and targets & Standard ML \\
Physics Loss & Newton-Euler equations (all 12 states) & [1, 2] \\
& - Euler rotational dynamics & [2] \\
& - Attitude kinematics & [2] \\
& - Translational dynamics (body frame) & [2] \\
& - Position kinematics (rotation matrices) & [2] \\
Temporal Smoothness & Derivative bounds (soft constraints) & [3] \\
Stability Loss & State-space bounds & [2] \\
Regularization & Normalized parameter penalty & [1] \\
\bottomrule
\end{tabular}
\caption{Mapping of loss function components to source citations}
\end{table}

\subsection{Complete Bibliography}

\begin{enumerate}
\item Raissi, M., Perdikaris, P., \& Karniadakis, G. E. (2019). Physics-informed neural networks: A deep learning framework for solving forward and inverse problems involving nonlinear partial differential equations. \textit{Journal of Computational Physics}, 378, 686-707.

\item Mahony, R., Kumar, V., \& Corke, P. (2012). Multirotor Aerial Vehicles: Modeling, Estimation, and Control of Quadrotor. \textit{IEEE Robotics \& Automation Magazine}, 19(3), 20-32.

\item Karniadakis, G. E., Kevrekidis, I. G., Lu, L., Perdikaris, P., Wang, S., \& Yang, L. (2021). Physics-informed machine learning. \textit{Nature Reviews Physics}, 3(6), 422-440.

\item Beard, R. W., \& McLain, T. W. (2012). \textit{Small Unmanned Aircraft: Theory and Practice}. Princeton University Press.
\end{enumerate}

\textbf{Note:} This minimal set of 4 sources provides complete coverage of all equation components in the loss function, from the foundational PINN framework to the specific quadrotor dynamics formulation.
