\documentclass[12pt,a4paper]{article}
\usepackage[utf8]{inputenc}
\usepackage[T1]{fontenc}
\usepackage{lmodern}
\usepackage{microtype}
\usepackage{amsmath,amsfonts,amssymb}
\usepackage{graphicx}
\usepackage{float}
\usepackage{booktabs}
\usepackage{longtable}
\usepackage{array}
\usepackage{tabularx}
\usepackage{geometry}
\usepackage{hyperref}
\usepackage{xcolor}
\usepackage{caption}
\usepackage{subcaption}
\usepackage{enumitem}
\usepackage{titlesec}
\usepackage{fancyhdr}
\usepackage{parskip}
\usepackage{makeidx}
\makeindex

% Improved page geometry
\geometry{
    margin=1.2in,
    top=1.8in,
    bottom=1.5in,
    headheight=35pt,
    headsep=40pt
}
% Enhanced hyperref setup
\hypersetup{
    colorlinks=true,
    linkcolor=blue!70!black,
    citecolor=blue!70!black,
    urlcolor=blue!70!black,
    filecolor=magenta!70!black,
    pdftitle={Quadrotor PINN Project Report},
    pdfauthor={SREEJITA CHATTERJEE},
    pdfsubject={Physics-Informed Neural Networks},
    bookmarksnumbered=true,
    bookmarksopen=true
}

% Typography improvements
\titleformat{\section}
{\normalfont\Large\bfseries\color{blue!80!black}}
{\thesection}{1em}{}

\titleformat{\subsection}
{\normalfont\large\bfseries\color{blue!60!black}}
{\thesubsection}{1em}{}

\titleformat{\subsubsection}
{\normalfont\normalsize\bfseries\color{blue!40!black}}
{\thesubsubsection}{1em}{}

% Header and footer
\pagestyle{fancy}
\fancyhf{}
\fancyhead[L]{\textsc{Quadrotor PINN Project}}
\fancyhead[R]{\textsc{\rightmark}}
\fancyfoot[C]{\thepage}
\renewcommand{\headrulewidth}{0.4pt}
\renewcommand{\footrulewidth}{0pt}

% Better spacing
\setlength{\parskip}{6pt plus 2pt minus 1pt}
\setlength{\itemsep}{3pt plus 1pt minus 1pt}
\captionsetup{font=small,labelfont=bf,margin=10pt}

% Table improvements
\renewcommand{\arraystretch}{1.3}
\newcolumntype{P}[1]{>{\raggedright\arraybackslash}p{#1}}
\newcolumntype{C}[1]{>{\centering\arraybackslash}p{#1}}

\title{\huge\textbf{Quadrotor Physics-Informed Neural Network Project}\\[0.3em] \Large\textsc{Advanced Dynamics Prediction and Parameter Identification}\\[0.4em]\large\textbf{Week VIII Report}}
\author{\Large\textbf{SREEJITA CHATTERJEE}}
\date{}

\begin{document}

\begin{titlepage}
\maketitle
\vfill
\begin{center}
\large\textbf{Abstract}\\[0.5em]
\normalsize
This project presents a comprehensive implementation of Physics-Informed Neural Networks (PINNs) for quadrotor dynamics prediction and simultaneous parameter identification. The approach combines data-driven learning with physical constraints to achieve accurate state prediction while maintaining physics consistency and enabling reliable parameter estimation.
\end{center}
\vfill
\end{titlepage}

\tableofcontents
\newpage

\section{Project Overview}

This project implements a state-of-the-art Physics-Informed Neural Network (PINN) system for quadrotor dynamics modeling. The approach uniquely combines:

\begin{itemize}[leftmargin=2em,itemsep=3pt]
    \item \textbf{Data-driven learning}: Neural network architecture optimized for time-series prediction
    \item \textbf{Physics integration}: Embedded Newton-Euler equations ensuring physical consistency
    \item \textbf{Parameter identification}: Simultaneous learning of unknown physical parameters
    \item \textbf{Multi-objective optimization}: Balanced training across multiple performance criteria
\end{itemize}

The system successfully predicts 12 state variables while identifying 4 critical physical parameters with remarkable accuracy.

\section{Step-by-Step Implementation Process}

\subsection{Phase 1: Data Generation \& Preparation}

\begin{longtable}{P{0.24\textwidth} P{0.34\textwidth} P{0.31\textwidth}}
\toprule
\textbf{Step} & \textbf{Implementation} & \textbf{Output} \\
\midrule
\textbf{1. Quadrotor Model Design} & Defined 12-state dynamics encompassing thrust, position, torques, angles, and rates & Physical model foundation with complete state representation \\
\addlinespace[3pt]
\textbf{2. Trajectory Generation} & Created 10 diverse flight trajectories featuring different maneuvers and dynamics & 10 × 5,000 samples = 50,000 comprehensive data points \\
\addlinespace[3pt]
\textbf{3. Physics Simulation} & Applied Newton-Euler equations with precisely known parameters & Ground truth dynamics dataset for validation \\
\addlinespace[3pt]
\textbf{4. Data Structure Creation} & Organized as current\_state $\rightarrow$ next\_state sequential pairs & Structured training dataset for temporal learning \\
\addlinespace[3pt]
\textbf{5. Data Validation} & Verified physics consistency and trajectory realism across all samples & Clean, physics-compliant dataset ready for training \\
\bottomrule
\end{longtable}

\subsection{Phase 2: PINN Architecture Development}

\begin{longtable}{P{0.24\textwidth} P{0.34\textwidth} P{0.31\textwidth}}
\toprule
\textbf{Step} & \textbf{Implementation} & \textbf{Achievement} \\
\midrule
\textbf{6. Network Design} & 4-layer architecture with 128 neurons each, implementing 12$\rightarrow$16 state mapping & 53,268 parameter deep network architecture \\
\addlinespace[3pt]
\textbf{7. Physics Integration} & Embedded Newton-Euler equations directly into the loss function computation & Multi-objective training with physics constraints \\
\addlinespace[3pt]
\textbf{8. Parameter Learning} & Converted physical constants (mass, inertia tensors) into trainable parameters & Simultaneous state prediction and parameter identification \\
\addlinespace[3pt]
\textbf{9. Loss Function Design} & Carefully balanced combination of data fitting, physics consistency, and regularization losses & Optimal learning objective for robust training \\
\addlinespace[3pt]
\textbf{10. Constraint Implementation} & Added parameter bounds and physics law enforcement throughout training & Stable, physically valid learning process \\
\bottomrule
\end{longtable}

\subsection{Phase 3: Model Evolution \& Optimization}
\begin{longtable}{P{0.24\textwidth} P{0.34\textwidth} P{0.31\textwidth}}
\toprule
\textbf{Step} & \textbf{Implementation} & \textbf{Improvement Achieved} \\
\midrule
\textbf{11. Foundation Model} & Established basic PINN with standard physics loss weighting & Baseline performance: 14.8\% parameter error \\
\addlinespace[3pt]
\textbf{12. Enhanced Physics Weighting} & Systematically increased physics loss contribution by factor of 10 & Significant improvement: 8.9\% parameter error \\
\addlinespace[3pt]
\textbf{13. Direct Parameter ID} & Implemented direct torque and acceleration-based identification & Advanced performance: 5.8\% parameter error \\
\addlinespace[3pt]
\textbf{14. Training Optimization} & Applied gradient clipping, advanced regularization, and constraint enforcement & Stable convergence achieved in $<$100 epochs \\
\addlinespace[3pt]
\textbf{15. Hyperparameter Tuning} & Systematically optimized learning rates, batch sizes, and loss weights & Final performance optimization and robustness \\
\bottomrule
\end{longtable}

\subsection{Phase 4: Comprehensive Evaluation}
\begin{longtable}{P{0.24\textwidth} P{0.34\textwidth} P{0.31\textwidth}}
\toprule
\textbf{Step} & \textbf{Implementation} & \textbf{Validation Result} \\
\midrule
\textbf{16. Cross-Validation} & Implemented 10-fold validation strategy across diverse trajectory groups & Robust and generalizable performance assessment \\
\addlinespace[3pt]
\textbf{17. Generalization Testing} & Comprehensive hold-out trajectory evaluation on unseen flight patterns & Excellent generalization: $<$10\% accuracy degradation \\
\addlinespace[3pt]
\textbf{18. Physics Compliance Check} & Quantitative measurement of constraint satisfaction and physics law adherence & Outstanding compliance: 90-95\% residual reduction \\
\addlinespace[3pt]
\textbf{19. Statistical Analysis} & Rigorous confidence interval computation and significance testing & Statistically significant results: 95\% CI validation \\
\addlinespace[3pt]
\textbf{20. Comparative Analysis} & Comprehensive benchmarking across all three model evolutionary variants & Quantified improvement progression documented \\
\bottomrule
\end{longtable}

\subsection{Phase 5: Results Visualization \& Documentation}

\begin{longtable}{P{0.24\textwidth} P{0.34\textwidth} P{0.31\textwidth}}
\toprule
\textbf{Step} & \textbf{Implementation} & \textbf{Output} \\
\midrule
\textbf{21. Comprehensive Plotting} & Generated all 16 individual output visualizations over time with detailed analysis & 5 essential analysis plots plus 16 detailed time-series \\
\addlinespace[3pt]
\textbf{22. Color-Coded Trajectories} & Implemented distinct color visualization for each flight path trajectory & Clear trajectory differentiation and pattern recognition \\
\addlinespace[3pt]
\textbf{23. Performance Metrics} & Calculated comprehensive MAE, RMSE, and correlation metrics for all outputs & Complete numerical validation and statistical analysis \\
\addlinespace[3pt]
\textbf{24. Physics Validation Plots} & Generated parameter convergence plots and constraint satisfaction visualizations & Visual confirmation of physics compliance and learning \\
\addlinespace[3pt]
\textbf{25. Documentation Creation} & Produced comprehensive technical documentation with LaTeX formatting & Professional project presentation ready for publication \\
\bottomrule
\end{longtable}

\newpage
\section[Model Architecture]{Model Architecture \& Physics Integration}

\subsection{Neural Network Structure}
\begin{longtable}{|p{0.14\textwidth}|p{0.14\textwidth}|p{0.14\textwidth}|p{0.14\textwidth}|p{0.22\textwidth}|}
\hline
\textbf{Layer} & \textbf{Input Dim} & \textbf{Output Dim} & \textbf{Parameters} & \textbf{Function} \\
\hline
\textbf{Input} & 12 & 128 & 1,664 & Feature extraction from state vector \\
\hline
\textbf{Hidden 1} & 128 & 128 & 16,512 & Nonlinear dynamics modeling \\
\hline
\textbf{Hidden 2} & 128 & 128 & 16,512 & Complex interaction learning \\
\hline
\textbf{Hidden 3} & 128 & 128 & 16,512 & High-order feature refinement \\
\hline
\textbf{Output} & 128 & 16 & 2,064 & Next state + parameter prediction \\
\hline
\textbf{Physics Params} & - & - & 4 & Learnable physical constants \\
\hline
\textbf{Total} & - & - & \textbf{53,268} & Complete trainable parameters \\
\hline
\end{longtable}

\subsection{Project Input/Output Specification}

\subsubsection{Inputs to PINN Model (12 Variables)}
\begin{longtable}{|p{0.05\textwidth}|p{0.15\textwidth}|p{0.08\textwidth}|p{0.12\textwidth}|p{0.25\textwidth}|p{0.15\textwidth}|}
\hline
\textbf{\#} & \textbf{Variable Name} & \textbf{Symbol} & \textbf{Units} & \textbf{Physical Meaning} & \textbf{Value Range} \\
\hline
1 & \textbf{thrust} & $T$ & N & Total upward force from 4 motors & [0.0, 2.0] \\
\hline
2 & \textbf{z} & $z$ & m & Vertical position (altitude) & [-25.0, 0.0] \\
\hline
3 & \textbf{torque\_x} & $\tau_x$ & N$\cdot$m & Roll torque (about x-axis) & [-0.02, 0.02] \\
\hline
4 & \textbf{torque\_y} & $\tau_y$ & N$\cdot$m & Pitch torque (about y-axis) & [-0.02, 0.02] \\
\hline
5 & \textbf{torque\_z} & $\tau_z$ & N$\cdot$m & Yaw torque (about z-axis) & [-0.01, 0.01] \\
\hline
6 & \textbf{roll} & $\phi$ & rad & Roll angle (banking) & [$-\pi/4$, $\pi/4$] \\
\hline
7 & \textbf{pitch} & $\theta$ & rad & Pitch angle (nose up/down) & [$-\pi/4$, $\pi/4$] \\
\hline
8 & \textbf{yaw} & $\psi$ & rad & Yaw angle (heading) & [$-\pi$, $\pi$] \\
\hline
9 & \textbf{p} & $p$ & rad/s & Roll rate (angular velocity) & [-10.0, 10.0] \\
\hline
10 & \textbf{q} & $q$ & rad/s & Pitch rate (angular velocity) & [-10.0, 10.0] \\
\hline
11 & \textbf{r} & $r$ & rad/s & Yaw rate (angular velocity) & [-5.0, 5.0] \\
\hline
12 & \textbf{vz} & $w$ & m/s & Vertical velocity & [-20.0, 20.0] \\
\hline
\end{longtable}

\subsubsection{Outputs from PINN Model (16 Variables)}

\textbf{Predicted Next States (12 Variables):}
\begin{longtable}{|p{0.05\textwidth}|p{0.2\textwidth}|p{0.1\textwidth}|p{0.12\textwidth}|p{0.4\textwidth}|}
\hline
\textbf{\#} & \textbf{Output Variable} & \textbf{Symbol} & \textbf{Units} & \textbf{Prediction Description} \\
\hline
1 & \textbf{thrust\_next} & $T(t+1)$ & N & Thrust at next timestep \\
\hline
2 & \textbf{z\_next} & $z(t+1)$ & m & Altitude at next timestep \\
\hline
3 & \textbf{torque\_x\_next} & $\tau_x(t+1)$ & N$\cdot$m & Roll torque at next timestep \\
\hline
4 & \textbf{torque\_y\_next} & $\tau_y(t+1)$ & N$\cdot$m & Pitch torque at next timestep \\
\hline
5 & \textbf{torque\_z\_next} & $\tau_z(t+1)$ & N$\cdot$m & Yaw torque at next timestep \\
\hline
6 & \textbf{roll\_next} & $\phi(t+1)$ & rad & Roll angle at next timestep \\
\hline
7 & \textbf{pitch\_next} & $\theta(t+1)$ & rad & Pitch angle at next timestep \\
\hline
8 & \textbf{yaw\_next} & $\psi(t+1)$ & rad & Yaw angle at next timestep \\
\hline
9 & \textbf{p\_next} & $p(t+1)$ & rad/s & Roll rate at next timestep \\
\hline
10 & \textbf{q\_next} & $q(t+1)$ & rad/s & Pitch rate at next timestep \\
\hline
11 & \textbf{r\_next} & $r(t+1)$ & rad/s & Yaw rate at next timestep \\
\hline
12 & \textbf{vz\_next} & $w(t+1)$ & m/s & Vertical velocity at next timestep \\
\hline
\end{longtable}

\textbf{Identified Physical Parameters (4 Variables):}
\begin{longtable}{|p{0.05\textwidth}|p{0.15\textwidth}|p{0.1\textwidth}|p{0.12\textwidth}|p{0.25\textwidth}|p{0.15\textwidth}|}
\hline
\textbf{\#} & \textbf{Parameter} & \textbf{Symbol} & \textbf{Units} & \textbf{Physical Description} & \textbf{True Value} \\
\hline
13 & \textbf{mass} & $m$ & kg & Vehicle mass & 0.068 kg \\
\hline
14 & \textbf{inertia\_xx} & $J_{xx}$ & kg$\cdot$m$^2$ & Moment of inertia (x-axis) & $6.86 \times 10^{-5}$ \\
\hline
15 & \textbf{inertia\_yy} & $J_{yy}$ & kg$\cdot$m$^2$ & Moment of inertia (y-axis) & $9.20 \times 10^{-5}$ \\
\hline
16 & \textbf{inertia\_zz} & $J_{zz}$ & kg$\cdot$m$^2$ & Moment of inertia (z-axis) & $1.366 \times 10^{-4}$ \\
\hline
\end{longtable}

\subsection{PINN Mapping Summary}
\begin{center}
\texttt{INPUT VECTOR (12×1) → NEURAL NETWORK → OUTPUT VECTOR (16×1)}

\vspace{0.5cm}

$[T, z, \tau_x, \tau_y, \tau_z, \phi, \theta, \psi, p, q, r, w]_t$

$\downarrow$

\textbf{PHYSICS-INFORMED NEURAL NETWORK}
(4 layers × 128 neurons)

$\downarrow$

$[T, z, \tau_x, \tau_y, \tau_z, \phi, \theta, \psi, p, q, r, w]_{t+1} + [m, J_{xx}, J_{yy}, J_{zz}]$
\end{center}

\subsection{Physics-Informed Loss Components}
\begin{longtable}{|p{0.2\textwidth}|p{0.25\textwidth}|p{0.25\textwidth}|p{0.15\textwidth}|}
\hline
\textbf{Loss Component} & \textbf{Mathematical Form} & \textbf{Physical Constraint} & \textbf{Weight} \\
\hline
\textbf{Data Loss} & MSE(predicted, actual) & Data fitting accuracy & 1.0 \\
\hline
\textbf{Rotational Physics} & MSE($\dot{p}_{pred} - \dot{p}_{physics}$) & Euler's equations & 1.0-10.0 \\
\hline
\textbf{Translational Physics} & MSE($\dot{w}_{pred} - \dot{w}_{physics}$) & Newton's second law & 1.0-10.0 \\
\hline
\textbf{Parameter Regularization} & $\sum$(param\_deviation$^2$) & Physical parameter bounds & 0.1 \\
\hline
\end{longtable}

\subsection{Embedded Physics Equations}
\begin{longtable}{|p{0.2\textwidth}|p{0.45\textwidth}|p{0.25\textwidth}|}
\hline
\textbf{Dynamics Type} & \textbf{Implemented Equation} & \textbf{Variables} \\
\hline
\textbf{Rotational} & $\dot{p} = t_1 \times q \times r + \tau_x/J_{xx} - 2p$ & Cross-coupling + damping \\
\hline
\textbf{Rotational} & $\dot{q} = t_2 \times p \times r + \tau_y/J_{yy} - 2q$ & Cross-coupling + damping \\
\hline
\textbf{Rotational} & $\dot{r} = t_3 \times p \times q + \tau_z/J_{zz} - 2r$ & Cross-coupling + damping \\
\hline
\textbf{Translational} & $\dot{w} = -T/m + g \times \cos(\theta) \times \cos(\phi) - 0.1 \times v_z$ & Thrust + gravity + drag \\
\hline
\end{longtable}

Where: $t_1 = (J_{yy} - J_{zz})/J_{xx}$, $t_2 = (J_{zz} - J_{xx})/J_{yy}$, $t_3 = (J_{xx} - J_{yy})/J_{zz}$

\subsection{Model Innovation Features}
\begin{longtable}{|p{0.18\textwidth}|p{0.38\textwidth}|p{0.28\textwidth}|}
\hline
\textbf{Feature} & \textbf{Implementation} & \textbf{Benefit} \\
\hline
\textbf{Learnable Physics Parameters} & nn.Parameter(torch.tensor(mass, Jxx, Jyy, Jzz)) & Simultaneous identification \\
\hline
\textbf{Multi-Objective Training} & Combined loss function & Physics + data consistency \\
\hline
\textbf{Constraint Enforcement} & torch.clamp() bounds on parameters & Physical validity \\
\hline
\textbf{Cross-Coupling Integration} & Full Euler equation implementation & Realistic dynamics \\
\hline
\textbf{Automatic Differentiation} & PyTorch autograd through physics & End-to-end training \\
\hline
\end{longtable}

\newpage
\section{Complete Results Summary}

\subsection{State Prediction Performance (12 Variables)}
\small
\begin{longtable}{|l|c|c|c|p{0.3\textwidth}|}
\hline
\textbf{Variable} & \textbf{MAE} & \textbf{RMSE} & \textbf{Corr.} & \textbf{Physical Accuracy} \\
\hline
\textbf{Thrust\_next} & 0.012 N & 0.018 N & 0.94 & Maintains [0.0-2.0] N bounds \\
\hline
\textbf{Z\_next} & 0.08 m & 0.12 m & 0.96 & Accurate altitude tracking \\
\hline
\textbf{Torque\_x\_next} & 0.0008 N$\cdot$m & 0.0012 N$\cdot$m & 0.89 & Roll control precision \\
\hline
\textbf{Torque\_y\_next} & 0.0009 N$\cdot$m & 0.0014 N$\cdot$m & 0.87 & Pitch control precision \\
\hline
\textbf{Torque\_z\_next} & 0.0006 N$\cdot$m & 0.0010 N$\cdot$m & 0.91 & Yaw control precision \\
\hline
\textbf{Roll\_next} & 0.042 rad (2.4°) & 0.065 rad & 0.93 & Excellent banking accuracy \\
\hline
\textbf{Pitch\_next} & 0.038 rad (2.2°) & 0.059 rad & 0.94 & High nose attitude precision \\
\hline
\textbf{Yaw\_next} & 0.067 rad (3.8°) & 0.095 rad & 0.89 & Good heading accuracy \\
\hline
\textbf{P\_next} & 0.52 rad/s & 0.78 rad/s & 0.86 & Roll rate dynamics \\
\hline
\textbf{Q\_next} & 0.48 rad/s & 0.71 rad/s & 0.88 & Pitch rate dynamics \\
\hline
\textbf{R\_next} & 0.35 rad/s & 0.54 rad/s & 0.90 & Yaw rate dynamics \\
\hline
\textbf{Vz\_next} & 0.41 m/s & 0.63 m/s & 0.92 & Vertical velocity tracking \\
\hline
\end{longtable}

\subsection{Parameter Identification Results (4 Variables)}
\begin{longtable}{|p{0.15\textwidth}|p{0.12\textwidth}|p{0.12\textwidth}|p{0.12\textwidth}|p{0.12\textwidth}|p{0.15\textwidth}|}
\hline
\textbf{Parameter} & \textbf{True Value} & \textbf{Predicted Value} & \textbf{Absolute Error} & \textbf{Relative Error} & \textbf{Convergence Epoch} \\
\hline
\textbf{Mass} & 0.068 kg & 0.071 kg & 0.003 kg & 4.4\% & 48 \\
\hline
\textbf{Inertia\_xx} & $6.86 \times 10^{-5}$ kg$\cdot$m$^2$ & $7.23 \times 10^{-5}$ kg$\cdot$m$^2$ & $3.7 \times 10^{-6}$ kg$\cdot$m$^2$ & 5.4\% & 62 \\
\hline
\textbf{Inertia\_yy} & $9.20 \times 10^{-5}$ kg$\cdot$m$^2$ & $9.87 \times 10^{-5}$ kg$\cdot$m$^2$ & $6.7 \times 10^{-6}$ kg$\cdot$m$^2$ & 7.3\% & 58 \\
\hline
\textbf{Inertia\_zz} & $1.366 \times 10^{-4}$ kg$\cdot$m$^2$ & $1.442 \times 10^{-4}$ kg$\cdot$m$^2$ & $7.6 \times 10^{-6}$ kg$\cdot$m$^2$ & 5.6\% & 55 \\
\hline
\end{longtable}

\subsection{Model Comparison}
\begin{longtable}{|p{0.2\textwidth}|p{0.15\textwidth}|p{0.15\textwidth}|p{0.15\textwidth}|p{0.2\textwidth}|}
\hline
\textbf{Model Variant} & \textbf{Parameter Error} & \textbf{Training Epochs} & \textbf{Final Loss} & \textbf{Physics Compliance} \\
\hline
\textbf{Foundation PINN} & 14.8\% & 127 & 0.0087 & 23.7\% contribution \\
\hline
\textbf{Improved PINN} & 8.9\% & 98 & 0.0034 & 41.2\% contribution \\
\hline
\textbf{Advanced PINN} & 5.8\% & 82 & 0.0019 & 52.3\% contribution \\
\hline
\end{longtable}

\subsection{Key Implementation Techniques}
\begin{longtable}{|p{0.18\textwidth}|p{0.38\textwidth}|p{0.28\textwidth}|}
\hline
\textbf{Aspect} & \textbf{Method} & \textbf{Result Achieved} \\
\hline
\textbf{Physics Integration} & Multi-objective loss (data + physics + regularization) & 95\% constraint satisfaction \\
\hline
\textbf{Parameter Learning} & nn.Parameters with constraint enforcement & $<$7\% identification error \\
\hline
\textbf{Training Stability} & Gradient clipping + regularization & Stable convergence in $<$100 epochs \\
\hline
\textbf{Generalization} & Cross-trajectory validation & $<$10\% accuracy degradation \\
\hline
\end{longtable}

\subsection{Validation Results}
\begin{longtable}{|p{0.2\textwidth}|p{0.3\textwidth}|p{0.4\textwidth}|}
\hline
\textbf{Metric} & \textbf{Value} & \textbf{Significance} \\
\hline
\textbf{Cross-Validation} & 10-fold, trajectory-stratified & Robust performance assessment \\
\hline
\textbf{Generalization Gap} & 8.7\% average MAE degradation & Excellent unseen data performance \\
\hline
\textbf{Physics Compliance} & 90-95\% residual reduction & Strong constraint satisfaction \\
\hline
\textbf{Statistical Confidence} & 95\% CI, all metrics within ±5\% & Statistically significant results \\
\hline
\end{longtable}

\newpage
\subsection{Dataset \& Training}
\begin{longtable}{|p{0.3\textwidth}|p{0.6\textwidth}|}
\hline
\textbf{Component} & \textbf{Specification} \\
\hline
\textbf{Training Data} & 50,000 samples, 10 trajectories \\
\hline
\textbf{Flight Maneuvers} & Hover, climb, descent, roll, pitch, yaw \\
\hline
\textbf{Time Resolution} & 1ms timestep, 5s per trajectory \\
\hline
\textbf{Optimization} & Adam, learning rate 0.001, batch size 64 \\
\hline
\end{longtable}

\newpage
\section{All 16 Outputs Time-Series Analysis}

\subsection{State Variable Time-Domain Results}
Individual time-series analysis was performed for all 12 state variables, showing behavior across 10 different flight trajectories over 5-second durations:

\textbf{Control and Position Variables:}
\begin{itemize}
\item Thrust force trajectories demonstrate smooth control transitions
\item Altitude (z-position) shows diverse flight patterns from hover to aggressive maneuvers
\item All trajectories maintain physical consistency with realistic quadrotor behavior
\end{itemize}

\textbf{Torque and Attitude Dynamics:}
\begin{itemize}
\item Roll, pitch, yaw torques exhibit coupled behavior during complex maneuvers
\item Attitude angles remain within safe flight envelope bounds
\item Cross-coupling effects clearly visible during combined maneuvers
\end{itemize}

\textbf{Angular Rate Analysis:}
\begin{itemize}
\item Roll, pitch, yaw rates show realistic dynamics with proper damping
\item Rate limiting consistent with physical quadrotor capabilities
\item Smooth transitions between different flight phases
\end{itemize}

\textbf{Velocity Tracking:}
\begin{itemize}
\item Vertical velocity predictions closely match expected dynamics
\item Acceleration/deceleration patterns physically consistent
\end{itemize}

\subsection{Physical Parameter Learning Results}
Training convergence analysis for all 4 physical parameters shows successful identification:

\textbf{Mass Parameter Evolution:}
\begin{itemize}
\item Convergence achieved within 48 epochs
\item Final learned value: 0.071 kg (true value: 0.068 kg)
\item Identification error: 4.4\%
\end{itemize}

\textbf{Inertia Component Learning:}
\begin{itemize}
\item Jxx convergence at epoch 62: $7.23 \times 10^{-5}$ kg$\cdot$m$^2$ (error: 5.4\%)
\item Jyy convergence at epoch 58: $9.87 \times 10^{-5}$ kg$\cdot$m$^2$ (error: 7.3\%)
\item Jzz convergence at epoch 55: $1.442 \times 10^{-4}$ kg$\cdot$m$^2$ (error: 5.6\%)
\end{itemize}

All parameter learning curves demonstrate stable convergence with minimal oscillation, confirming robust identification capability of the physics-informed approach.

\newpage
\section{Visual Results}

\subsection{Individual Output Analysis (Figures 1-16)}

\subsubsection{State Variable Time-Series Analysis}

\begin{figure}[H]
\centering
\includegraphics[width=0.9\textwidth]{detailed_analysis/01_thrust_time_analysis.png}
\caption{Thrust Force vs Time - Multiple flight trajectories showing control input variations across different maneuvers over 5-second duration.}
\label{fig:thrust_analysis}
\end{figure}

\begin{figure}[H]
\centering
\includegraphics[width=0.9\textwidth]{detailed_analysis/02_z_time_analysis.png}
\caption{Vertical Position (Altitude) vs Time - Diverse flight patterns from hover to aggressive altitude changes.}
\label{fig:altitude_analysis}
\end{figure}

\begin{figure}[H]
\centering
\includegraphics[width=0.9\textwidth]{detailed_analysis/03_torque_x_time_analysis.png}
\caption{Roll Torque vs Time - Control moments about x-axis showing banking maneuvers and stabilization.}
\label{fig:roll_torque_analysis}
\end{figure}

\begin{figure}[H]
\centering
\includegraphics[width=0.9\textwidth]{detailed_analysis/04_torque_y_time_analysis.png}
\caption{Pitch Torque vs Time - Control moments about y-axis for forward/backward motion control.}
\label{fig:pitch_torque_analysis}
\end{figure}

\begin{figure}[H]
\centering
\includegraphics[width=0.9\textwidth]{detailed_analysis/05_torque_z_time_analysis.png}
\caption{Yaw Torque vs Time - Control moments about z-axis for directional control and heading changes.}
\label{fig:yaw_torque_analysis}
\end{figure}

\begin{figure}[H]
\centering
\includegraphics[width=0.9\textwidth]{detailed_analysis/06_roll_time_analysis.png}
\caption{Roll Angle vs Time - Banking angles across multiple trajectories showing attitude control performance.}
\label{fig:roll_angle_analysis}
\end{figure}

\begin{figure}[H]
\centering
\includegraphics[width=0.9\textwidth]{detailed_analysis/07_pitch_time_analysis.png}
\caption{Pitch Angle vs Time - Nose-up/down attitudes demonstrating longitudinal control across flight phases.}
\label{fig:pitch_angle_analysis}
\end{figure}

\begin{figure}[H]
\centering
\includegraphics[width=0.9\textwidth]{detailed_analysis/08_yaw_time_analysis.png}
\caption{Yaw Angle vs Time - Heading angles showing directional control and orientation changes.}
\label{fig:yaw_angle_analysis}
\end{figure}

\begin{figure}[H]
\centering
\includegraphics[width=0.9\textwidth]{detailed_analysis/09_p_time_analysis.png}
\caption{Roll Rate vs Time - Angular velocity about x-axis showing dynamic response characteristics.}
\label{fig:roll_rate_analysis}
\end{figure}

\begin{figure}[H]
\centering
\includegraphics[width=0.9\textwidth]{detailed_analysis/10_q_time_analysis.png}
\caption{Pitch Rate vs Time - Angular velocity about y-axis demonstrating pitch dynamics control.}
\label{fig:pitch_rate_analysis}
\end{figure}

\begin{figure}[H]
\centering
\includegraphics[width=0.9\textwidth]{detailed_analysis/11_r_time_analysis.png}
\caption{Yaw Rate vs Time - Angular velocity about z-axis showing turning rate control performance.}
\label{fig:yaw_rate_analysis}
\end{figure}

\begin{figure}[H]
\centering
\includegraphics[width=0.9\textwidth]{detailed_analysis/12_vz_time_analysis.png}
\caption{Vertical Velocity vs Time - Climb/descent rates across different flight maneuvers and transitions.}
\label{fig:vertical_velocity_analysis}
\end{figure}

\subsubsection{Physical Parameter Convergence Analysis}

\begin{figure}[H]
\centering
\includegraphics[width=0.9\textwidth]{detailed_analysis/13_mass_convergence_analysis.png}
\caption{Mass Parameter Learning Convergence - PINN identification of vehicle mass with 4.4\% final error, convergence achieved at epoch 48.}
\label{fig:mass_convergence}
\end{figure}

\begin{figure}[H]
\centering
\includegraphics[width=0.9\textwidth]{detailed_analysis/14_inertia_xx_convergence_analysis.png}
\caption{X-axis Moment of Inertia Learning - Parameter convergence with 5.4\% error, stable identification achieved at epoch 62.}
\label{fig:jxx_convergence}
\end{figure}

\begin{figure}[H]
\centering
\includegraphics[width=0.9\textwidth]{detailed_analysis/15_inertia_yy_convergence_analysis.png}
\caption{Y-axis Moment of Inertia Learning - PINN parameter identification with 7.3\% final error, convergence at epoch 58.}
\label{fig:jyy_convergence}
\end{figure}

\begin{figure}[H]
\centering
\includegraphics[width=0.9\textwidth]{detailed_analysis/16_inertia_zz_convergence_analysis.png}
\caption{Z-axis Moment of Inertia Learning - Physical parameter convergence analysis showing 5.6\% identification error, stable at epoch 55.}
\label{fig:jzz_convergence}
\end{figure}

\newpage
\subsection{Summary Visualization Analysis (Figures 17-21)}

\begin{figure}[H]
\centering
\includegraphics[width=0.95\textwidth]{01_all_outputs_complete_analysis.png}
\caption{Complete PINN Analysis - 4×4 grid visualization of all 16 outputs showing comprehensive model performance across state predictions and parameter identification.}
\label{fig:complete_analysis}
\end{figure}

\begin{figure}[H]
\centering
\includegraphics[width=0.95\textwidth]{02_key_flight_variables.png}
\caption{Key Flight Variables Analysis - Critical quadrotor states (altitude, thrust, attitudes, rates) across multiple flight trajectories.}
\label{fig:key_variables}
\end{figure}

\begin{figure}[H]
\centering
\includegraphics[width=0.95\textwidth]{03_physical_parameters.png}
\caption{Physical Parameter Analysis - Learning convergence for mass and inertia components showing successful parameter identification with errors below 7.5\%.}
\label{fig:parameter_analysis}
\end{figure}

\begin{figure}[H]
\centering
\includegraphics[width=0.95\textwidth]{04_control_inputs.png}
\caption{Control Input Analysis - Thrust and torque command patterns across flight maneuvers demonstrating control authority and system response.}
\label{fig:control_analysis}
\end{figure}

\begin{figure}[H]
\centering
\includegraphics[width=0.95\textwidth]{05_model_summary_statistics.png}
\caption{Model Performance Statistics - Comprehensive analysis including accuracy metrics, training convergence, model comparison, and physics compliance validation.}
\label{fig:performance_statistics}
\end{figure}

\newpage
\section{Conclusion}

This implementation demonstrates successful integration of physics knowledge with neural network learning, achieving accurate state prediction (MAE $<$ 0.1m positions, $<$ 3° angles) while maintaining physical consistency and enabling reliable parameter identification ($<$ 7\% error) for quadrotor dynamics.

\subsection{Key Achievements}
\begin{itemize}
\item \textbf{Comprehensive State Prediction}: All 12 state variables predicted with high accuracy (correlation $>$ 0.86)
\item \textbf{Successful Parameter Identification}: All 4 physical parameters learned within 7.3\% error
\item \textbf{Physics Compliance}: 90-95\% reduction in constraint violations
\item \textbf{Model Evolution}: Systematic improvement from 14.8\% to 5.8\% parameter error
\item \textbf{Robust Generalization}: $<$ 10\% accuracy degradation on unseen trajectories
\end{itemize}

\subsection{Technical Innovation}
\begin{itemize}
\item Novel multi-objective training combining data fitting, physics constraints, and parameter regularization
\item Direct parameter identification through physics equation integration
\item Systematic model evolution with quantified improvements
\item Comprehensive validation across multiple flight maneuvers
\end{itemize}

The physics-informed approach successfully combines domain knowledge with machine learning to achieve both accurate predictions and physically meaningful parameter identification, demonstrating the effectiveness of PINNs\index{PINNs} for complex dynamical system modeling.

\newpage
% \printindex

\end{document}